\documentclass[10pt]{article}
\usepackage[english]{babel}
\usepackage{amsmath,amssymb,amsthm}
\usepackage[shortlabels]{enumitem}

\title{Problem set 10}
\author{Pablo Boixeda Alvarez}

\newtheorem{Prob}{Problem}
\newtheorem{Prop}{Proposition}
\newtheorem{Opt}{Optional Problem}

\theoremstyle{definition}
\newtheorem{Not}{Notation}
\newtheorem{Def}{Definition}
\newtheorem{eg}{Example}
\newtheorem{re}{Reading}

\theoremstyle{remark}
\newtheorem{rmk}{Remark}

\newenvironment{solution}
  {\renewcommand\qedsymbol{$\blacksquare$}\begin{proof}[Solution]}
  {\end{proof}}

\begin{document}
\section*{Math 350 Introduction to Abstract algebra}
\subsection*{Problem set 10}


\begin{re}
	DF 4.5, 5.1-5.2
\end{re}
\begin{Prob} (Starred* problems are required)
	
	DF 7.1: 3*, 4, 6, 7*, 13, 14* (Hint: $(1+x)(1-x)=1-x^2$ will help you if $x^2=0$, what do you do if $x^n=0$?), 15, 21*, 30*
	
	DF 7.2: 3*
	
\end{Prob}

\begin{solution}
\textbf{7.1.3} WLOG, assume ring $R$ and subring $S$ and let unit $u \in S$. Then, by definition, there exists $v \in S$ such that $vu = 1_S$. However, note that $vu = 1_S = 1_R$ and that $u,v \in R$ since $S \leq R$. Thus, it follows that $u$ is also a unit in $R$.

For the converse, assume that $R = \mathbb{Q}$ and subring $S = \mathbb{Z}$ and that $u = 2$ is a unit in $R$. Indeed for $v = \frac{1}{2}$, $uv = 1$, but since $v \notin \mathbb{Z}$, $u$ is not a unit in the subring $S$, so the converse does not hold true.

\end{solution}

\begin{solution}
\textbf{7.1.7} First, note that $Z(R)$ is nonempty since $0 \in Z(R)$ as $0 \cdot r = r \cdot 0$. Then, for $x,y \in Z(R)$ and $r \in R$, $(x-y)r = xr-yr = rx-ry = r(x-y)$. Thus, it follows that the center is a subring. Assuming $R$ has 1, $1 \cdot x = x \cdot 1$ for all $x \in R$ implies that $1 \in Z(R)$.

Next, suppose that $R$ is a divison ring with center $Z(R)$. By definition, every $x$ has a unique inverse $x^{-1}$. Since $\times$ is associative, $1 = a(bb^{-1})a^{-1} = (ab)(a^{-1}b^{-1})$. Thus, $(ab)^{-1} = b^{-1}a^{-1}$. Then, for $r \in R$, we have $x^{-1}r^{-1} = (rx)^{-1} = (xr)^{-1} = x^{-1}r^{-1}$, so $x^{-1} \in Z(R)$ is commutative for a division ring, making it a field.
\end{solution}

\begin{solution}
\textbf{7.1.14} \begin{enumerate}[a)]
    \item Assume nilpotent $x$, so $x^m = 0$ where $m \in \mathbb{N}$. If $m = 1$, then clearly $x = 0$. If $m > 1$, $x \neq 1$ since equality would yield a contradiction since a smaller $m$ exists. Thus, we have  $x^m = x \cdot x^{m-1} = 0$, which clearly implies that $x$ is a zero divisor.
    
    \item Since $R$ is commutative, we have that $(rx)^m = r^mx^m = r^m \cdot 0 = 0$ for $r \in R$, so $rx$ is clearly nilpotent.
    
    \item Suppose that $x$ is nilpotent with $m \in R$. Then, $1 -(-1)^mx^m = 1$. Note that we can factor the LHS as $1 -(-1)^mx^m = (1+x)(1-x+x^2-...+(-1)^{m-1}x^{m-1}) = 1$. Thus, we can clearly see that $1+x$ is a unit since the term $\sum_{i=0}^{m-1}(-x)^i \in R$.
    
    \item Let $u$ be a unit with $uv = 1$ and $x$ be nilpotent. Then, $u^{-1}x$ is nilpotent and $u^{-1}x + 1$ is a unit with $(u^{-1}x + 1)w = 1$. Thus, we have $u(u^{-1}x + 1) = x + u$ is a unit as well since $u(u^{-1}x + 1) \cdot vw = 1$.
\end{enumerate}
\end{solution}

\begin{solution}
\textbf{7.1.21}
\begin{enumerate}[a)]
    \item Let $A, B, C \subseteq X$. Let's first prove associativity.
    
    \begin{align*}
        (A+B)+C &= ((A-B) \cup (B-A))+C \\
                &= (((A-B)\cup(B-A))-C) \cup (C-((A-B)\cup(B-A)))
    \end{align*}
    
    \item Since $A \times B = A \cap B = B \cap A = B \times A$, $R$ is clearly commutative. Let $A \subseteq X \subseteq R$. Then, $A \times X = A \cap X = A$ and $X \times A = X \cap A = A$, so $R$ has an identity. Furthermore, for all $A \in R$, $A \times A = A \cap A = A$, so $R$ is a Boolean ring. 
\end{enumerate}
\end{solution}

\begin{solution}
\textbf{7.1.30}
\end{solution}

\begin{solution}
\textbf{7.2.3}
\begin{enumerate}[a)]
    \item Let's first show that $R[[x]]$ is a ring. We can clearly see that $(R[[x]],+)$ form an abelian group since if we group the same powers of $x$, we are just simply just adding real or complex coefficients, which we know both form a group. Moreover, the addition operator is also know to be abelian under the reals and complex numbers, so $(R[[x]],+)$ is an abelian group. Next, consider $\alpha = \sum_{i=0}^\infty a_nx^n, \beta = \sum_{i=0}^\infty b_nx^n, \gamma = \sum_{i=0}^\infty c_nx^n$.
    
    \begin{align*}
        (\alpha \cdot \beta) \cdot \gamma &= (\sum_{i=0}^\infty a_nx^n \cdot \sum_{i=0}^\infty b_nx^n) \cdot \sum_{i=0}^\infty c_nx^n \\
                                          &=  (\sum_{n \geq 0}(\sum_{i+j = n}a_ib_j)x^n)(\sum_{n \geq 0} c_nx^n)\\
                                          &= \sum_{n \geq 0}(\sum_{t+k = n}(\sum_{i+j=t} a_ib_j)c_k)x^n \\
                                          &= \sum_{n \geq 0}(\sum_{i+j+k = n}a_ib_jc_k)x^n \\
                                          &= \sum_{n \geq 0}(\sum_{i+s=n}a_i(\sum_{j+k=s}b_jc_k))x^n \\
                                          &= (\sum_{n \geq 0} a_nx^n)(\sum_{n \geq 0}(\sum_{j+k=n}b_jc_k)x^n) \\
                                          &= (\sum_{n \geq 0} a_nx^n)(((\sum_{n \geq 0} b_nx^n)(\sum_{n \geq 0} c_nx^n)) \\
                                          &= \alpha ( \beta \cdot \gamma) \\
        (\alpha + \beta) \cdot \gamma &= (\sum_{n \geq 0} a_nx^n + \sum_{n \geq 0} b_nx^n) \cdot \sum_{n \geq 0} c_nx^n \\
                                      &= (\sum_{n \geq 0} (a_n+b_n)x^n) \cdot \sum_{n \geq 0} c_nx^n \\
                                      &= \sum_{n \geq 0} (\sum_{i+j=n} (a_i+b_i)c_j) x^n \\
                                      &= \sum_{n \geq 0} (\sum_{i+j=n} (a_ic_j + b_ic_j) x^n \\
                                      &= \sum_{n \geq 0} (\sum_{i+j=n} a_ic_j) x^n + \sum_{n \geq 0} (\sum_{i+j=n} b_ic_j) x^n\\
                                      &= (\sum_{n \geq 0} a_nx^n)(\sum_{n \geq 0} c_nx^n) + (\sum_{n \geq 0} b_nx^n)(\sum_{n \geq 0} c_nx^n) \\
                                      &= \alpha \cdot \gamma + \beta \cdot \gamma \\
    \end{align*}
    Thus, $R[[x]]$ is a ring, and now we prove commutativity of multiplication.
    
    \begin{align*}
        \alpha \cdot \beta &= (\sum_{n \geq 0} a_nx^n) \cdot (\sum_{n \geq 0} b_nx^n) \\
                           &= \sum_{n \geq 0}(\sum_{i+j = n} a_ib_j) x^n \\
                           &= \sum_{n \geq 0}(\sum_{i+j = n} b_ja_i) x^n \\
                           &= (\sum_{n \geq 0} b_nx^n) \cdot (\sum_{n \geq 0} a_nx^n) \\
                           &= \beta \cdot \alpha
    \end{align*}
    
    Furthermore, we can easily show the identity by $\sum_{i=0}^\infty a_nx^n$ where $a_0 = 1$ and $a_i = 0$ for $i > 0$. Indeed, we have shown that $R[[x]]$ is a commutative ring with 1.
    
    \item Multiplying the two terms together, we get $(1-x) \cdot (\sum_{n \geq 0}x^n) = \sum_{n \geq 0}x^n - \sum_{n \geq 0}x^{n+1} = 1 + \sum_{n \geq 1}x^n - x^n = 1$. Thus, we can clearly see that $1-x$ is a unit.
    
    \item $(\rightarrow)$ Suppose $\sum_{n \geq 0}a_nx^n$ is a unit in $R[[x]]$ such that $(\sum_{n \geq 0} a_nx^n) \cdot (\sum_{n \geq 0} b_nx^n) = \sum_{n \geq 0}(\sum_{i+j = n} a_ib_j) x^n = 1$. Clearly $a_0b_0 = 1$ here, so $a_0$ is a unit. 
    
    $(\leftarrow)$ Conversely, suppose that $a_0$ is a unit in $R$. Then, define it's inverse as $a_0b_0 = 1$. Next, define $b_{n+1} = -b_0 \sum_{i+j=k+1, j \leq n}a_ib_j$. Then, for $(\sum_{n \geq 0} a_nx^n) \cdot (\sum_{n \geq 0} b_nx^n) = \sum_{n \geq 0}(\sum_{i+j = n} a_ib_j) x^n$, we have for $n \geq 1$,
    
    \begin{align*}
        \sum_{i+j=n}a_ib_j &= a_0b_n + \sum_{i+j=n, j<n}a_ib_j \\
                           &= a_0(-b_0 \sum_{i+j=n, j < n}a_ib_j) + \sum_{a+j=n, j<n}a_ib_j \\
                           &= 0 \\
    \end{align*}
    
    Thus, we have that $(\sum_{n \geq 0} a_nx^n) \cdot (\sum_{n \geq 0} b_nx^n) = 1 + 0 + 0 + ... = 1$, so we are done.
    
\end{enumerate}

\end{solution}


\begin{Prob}
	\underline{Symmetric polynomials}: Let $R$ be a commutative ring with $1\neq 0$.
	\begin{enumerate}[a)]
		\item Consider the symmetric group $S_n$ acting on the set $\{x_1,\dots, x_n\}$ by permutations. We can extend this to an action on $R[x_1, x_2, \dots, x_n]$. For example, if $\sigma = (123) \in S_3$, then
		$$\sigma \cdot (x_1x_2 - 2x^2_3 + 3x_2x^2_3) = x_2x_3 - 2x^2_1 + 3x_3 x^2_1.$$
		Prove that this action satisfies $\sigma \cdot (f + g) = \sigma \cdot f + \sigma \cdot g$ and $\sigma \cdot (fg) = (\sigma \cdot f)(\sigma \cdot g)$ for all $\sigma \in S_n$ and all $f, g \in R[x_1, \dots , x_n]$. (Hint: Consider monomials.)
		\item Let $S \subset R[x_1, \dots , x_n]$ be the set of multivariable polynomials that are fixed under the action of $S_n$. Prove that $S$ is a subring with $1$. This is called the ring of symmetric polynomials.
		\item For each $n \geq 0$, define polynomials $e_i \in R[x_1, \dots , x_n]$ by $e_0 = 1$ and $$e_1 = x_1 + \dots + x_n, e_2 =\sum_{1\leq i<j\leq n} x_ix_j , \dots , e_n = x_1 \dots x_n$$
		and $e_k = 0$ for $k > n$. In words, $e_k$ is the sum of all distinct products of subsets of $k$ distinct variables. Prove that each $e_k$ is a symmetric polynomial. These are called the elementary symmetric polynomials.
		\item The generic polynomial of degree $n$ is the polynomial
		$$f(x) = (x - x_1)(x - x_2)\dots (x - x_n)$$
		in the ring $R[x_1, \dots , x_n][x]$ of polynomials in $x$ with coefficients in $R[x_1, \dots , x_n]$. Prove (by induction) that
		$$f(x) = (x-x_1)(x-x_2)\dots(x-x_n) = x^n -e_1x^{n-1}+e_2x^{n-2}+\dots+ (-1)^n e_n =\sum_{j=0}^{n}(-1)^{n-j}e_{n-j}x^j.$$
		\item For each $k \geq 1$, define the power sums $p_k = x^k_1+\dots+x^k_n\in R[x_1, \dots , x_n]$. Clearly, the
		power sums are symmetric. Verify the following identities by hand:
		$$p_1 = e_1, p_2 = e_1p_1 - 2e_2, p_3 = e_1p_2 - e_2p_1 + 3e_3$$
		In general Newton’s identities in $R[x_1, . . . , x_n]$ are (recall that $e_k = 0$ for $k > n$):
		$$p_k - e_1p_{k-1} + e_2p_{k-2} - \dots + (-1)^{k-1}e_{k-1}p_1 + (-1)^k ke_k = 0.$$
		Prove Newton’s identities whenever $k \geq n$.
		(Hint: For each $i$, consider the equation in part (d) for $f(x_i)$ and sum all these equations
		together. This gives Newton’s identity for $k = n$. Set extra variables to zero to get the
		identities for $k > n$ from this.) (Optional: Can you come up with a proof when $1 \leq k \leq n$?)

		
	\end{enumerate} 
\end{Prob}

\begin{solution}
    \begin{enumerate}[a)]
    \item hello
    \item bye
    
    \end{enumerate}
\end{solution}


\footnotetext{Email address: pablo.boixedaalvarez@yale.edu}



\end{document}